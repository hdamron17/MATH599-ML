\documentclass{article}

\usepackage[bib,bibstyle=numeric,smalltitle,code]{shorthand}
\usepackage{csvsimple}
\usepackage{subfig}

\addbibresource{Report.bib}

\setcounter{secnumdepth}{0}

\title{Principal Component Analysis Report}
\author{Hunter Damron}
\date{21 March 2019}

\csvset{
	table head=\hline\csvlinetotablerow\\
}

\begin{document}
	\maketitle

	\section{Introduction}

	\section{Procedure}

	\section{Results}
		\begin{table}[!htbp]
			\centering
			\caption{PCA on 2-dimensional dataset\vspace{-3ex}}
			\begin{tabularx}{0.9\linewidth}{YYY}
				\subfloat{\csvautotabular[table head=\hline\csvlinetotablerow\\]{data/dataset2d.csv}}
				&
				\subfloat{\csvautotabular[table head=\hline\csvlinetotablerow\\]{output/dataset2d-reduced-1.csv}}
				&
				\subfloat{\csvautotabular[table head=\hline\csvlinetotablerow\\]{output/dataset2d-restored-1.csv}}
				\\\rule[1ex]{0pt}{0pt}
				Original Dataset
				&
				Reduced to 1 Dimension
				&
				Restored from 1 Dimension
			\end{tabularx}
		\end{table}

	\section{Discussion}

	\section{External Use of PCA}

	\nocite{*}
	\printbibliography{}

	\appendix
	\section{Appendix}
	\setcounter{secnumdepth}{2}
	\renewcommand{\thesubsection}{\Alph{subsection}}
	\subsection{PCA Code}
	\lstinputlisting{PCA.py}
	\subsection{Dataset Generation}
	\lstinputlisting{gendataset.py}
\end{document}
